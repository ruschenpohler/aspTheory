\documentclass[11.5pt]{article}

\usepackage{amssymb, amsmath, amsthm, amsfonts, eurosym, geometry, ulem, graphicx, caption, color, setspace, footmisc, pdflscape, array, todonotes, subfig, adjustbox, booktabs, pdfpages, siunitx, setspace, multirow, tikz, appendix}

\usepackage[flushleft]{threeparttable}
%\usepackage{hyperref}
\usepackage{natbib}
\bibliographystyle{plainnat}
%\usepackage[round]{natbib}
\usepackage[applemac]{inputenc}
\usepackage{hyperref}
%\usepackage{xcolor}
%\hypersetup{
%  colorlinks   = true, %Colours links instead of ugly boxes
%  urlcolor     = red, %Colour for external hyperlinks
%  linkcolor    = blue, %Colour of internal links
%  citecolor   = blue %Colour of citations
%}

\geometry{left=1in,right=1in,top=1in,bottom=1in}

\captionsetup{justification=justified}

\setlength{\textwidth}{6in}
\setlength{\oddsidemargin}{.22in}
\setlength{\textheight}{9in}
\setlength{\topmargin}{-0.10in}
\setlength{\headheight}{0.02in}

\normalem

\onehalfspacing
\newtheorem{theorem}{Theorem}
\newtheorem{corollary}[theorem]{Corollary}
\newtheorem{proposition}{Proposition}

\newtheorem{conjecture}{Conjecture}
\newtheorem{example}{Example}
\newtheorem{definition}{Definition}
\newtheorem{assumption}{Assumption}
\newtheorem{remark}{Remark}

%\newenvironment{proof}[1][Proof]{\noindent\textbf{#1.} }{\ \rule{0.5em}{0.5em}}

\newtheorem{hyp}{Hypothesis}
\newtheorem{subhyp}{Hypothesis}[hyp]
\renewcommand{\thesubhyp}{\thehyp\alph{subhyp}}

\newcommand{\red}[1]{{\color{red} #1}}
\newcommand{\blue}[1]{{\color{blue} #1}}

\newcolumntype{L}[1]{>{\raggedright\let\newline\\arraybackslash\hspace{0pt}}m{#1}}
\newcolumntype{C}[1]{>{\centering\let\newline\\arraybackslash\hspace{0pt}}m{#1}}
\newcolumntype{R}[1]{>{\raggedleft\let\newline\\arraybackslash\hspace{0pt}}m{#1}}

\geometry{left=1.0in,right=1.0in,top=1.0in,bottom=1.0in}

\captionsetup{justification=justified}
\renewcommand{\baselinestretch}{1.3}
\newcounter{marginparcounter}
\newcommand{\patopar}[1]{\stepcounter{marginparcounter}$^{(\roman{marginparcounter})}$\marginpar{\color{red}\renewcommand{\baselinestretch}{0.8}\scriptsize$^{(\roman{marginparcounter})}$ PD: #1}}
\renewcommand{\baselinestretch}{1.3}
\setlength{\textwidth}{6in}
\setlength{\oddsidemargin}{.22in}
\setlength{\textheight}{9in}
\setlength{\topmargin}{-0.10in}
\setlength{\headheight}{0.02in}
%\setlength{\parskip}{0.4cm}

\renewcommand{\thesection}{\Roman{section}}
\renewcommand{\thesubsection}{\Alph{subsection}}

\newcommand{\appendixpagenumbering}{
  \break
  \pagenumbering{arabic}
  \renewcommand{\thepage}{\thesection-\arabic{page}}
}

% ---------------------------------------------------------------------------------------
% ---------------------------------------------------------------------------------------

\begin{document}


\thispagestyle{empty}

\pagebreak

\onehalfspacing

\setcounter{page}{1}


\section{Contribution to the Literature} \label{sec:cont}

This paper contributes to several strands of literature. First, it adds to the empirical literature on aspirations and poverty (e.g., Bernard et al., 2014; Beaman et al., 2012; Janzen et al., 2017). This literature has largely corroborated the view that aspirations correlate with forward-looking behaviour and investment using cross-sectional or panel data (see, e.g., Janzen et al., 2017; Dalton et al., 2018; Kosec and Mo, 2017; Favara, 2017; Ross, 2017; Serneels and Dercon, 2014). Our findings contribute to this literature by showing the causal implications of aspirations. Moreover, while some recent work suggests aspirations may indeed be amenable to change by exposure to role models (see, e.g., Bernard et al., 2014; Macours and Vakis, 2014; Beaman et al., 2012; Riley, 2018), the evidence has been largely confined to educational aspirations. It is unclear whether the same holds for small-scale businesses, a domain of great relevance for the developing world (Maloney 2004; Gollin 2008; Nichter and Goldmark 2009). Further, a key notion of Ray (2006) and Genicot and Ray (2017), that aspirations which far exceed the individual's potential may result in aspirations frustration and lower effort, has been studied only with cross-sectional data (see, e.g., Janzen et al., 2017). In this paper, we provide first experimental evidence on the impact of a widening in aspirations windows on both business aspirations and business performance. On a more general level, by promoting the use of simple business practices that can be implemented at no monetary cost, we differ from Bernard et al. (2014) and Galiani et al. (2018) in that we show the effects of aspirations in an environment where economic constraints to satisfying higher aspirations are not binding. We add further by investigating the important role of providing soft psychological and implementation nudges to feed the entrepreneurs' capacity to aspire and to achieve. %Finally, we see the results on heterogeneous treatment effects as being the first experimental confirmation for the theoretical predictions of Ray (2006), Genicot and Ray (2017), and Dalton et al (2016). % Bernard et al were first..
Finally, we contribute methodologically in providing a novel approach to measuring aspirations on the essential dimensions of small-business growth.

Second, in providing evidence of the entrepreneur's aspirations for both their business and their children's educational prospects, we also contribute to a nascent literature on the multidimensionality of aspirations (see, e.g., Bernard et al., 2014; Bjorvatn et al., 2015), by providing first experimental evidence on the effect of a widening of business aspirations windows on multiple aspirations dimensions. Ray (2003, 2006) acknowledges that aspirations may best be seen as multi-faced in nature and that ``depending on one's place in the socio-economic hierarchy, these many-faceted aspirations may complement one another, or they may be mutual substitutes'' (Ray, 2003, p.2). While the empirical literature is still sparse, there is some recent evidence on the interplay of different aspirations dimensions. Bernard et al. (2014) report treatment effects of a role-model intervention among Ethiopian villagers on the aspirations towards their children's schooling, but no changes in income aspirations. While the authors conjecture that the finding may be due to a strong local belief in the returns to education, the aspirations dimensions in question may simply act as substitutes rather than complements. In that view, an exogenous shock may render salient one dimension at the expense of another in the pursuit of a goal or change the relative marginal benefits. Bjorvatn et al. (2015) offers further suggestive evidence along these lines from a field experiment among school students in Tanzania. The authors show that exposure to an edutainment program that motivated students for entrepreneurship had the effect of facilitating interest in entrepreneurship and to increase rates of business start-up but decreased school performance and graduation rates. Here, we explore the possibility that similar substitution effects occur in the domain of business aspirations. This is particularly important in a context where large parts of self-employment are essentially subsistence-oriented. Our findings suggest that business and family aspirations of the small-scale entrepreneurs in our sample are complements rather than substitute.

Third, this paper adds to the literature on aspirations and well-being (see, e.g., Stutzer, 2004; Knight and Gunatilaka, 2012; Bernard et al., 2014). While lifting aspirations may have a positive impact on effort and, in turn, achievement, it is not clear whether such gains translate into higher levels of overall subjective well-being. In fact, Stutzer (2004) and Knight and Gunatilaka (2012) provide evidence to the contrary: in large cross-sections from Switzerland and rural China, both show income aspirations to be negatively correlated with individual life satisfaction. However, both studies measure aspirations as minimum income needs rather than as an expression of preference over future states given unconstrained pursuit as has become the convention of the field (see, e.g., Ray, 2006; Genicot and Ray, 2017; Dalton et al., 2016; Bernard et al, 2014; Riley, 2017; Janzen et al., 2017; Dalton et al., 2018). Self-reported happiness and satisfaction have been used as proxies for individual utility in the past (see, e.g., Clark and Oswald, 1994; Oswald, 1997; Ng, 1997; Easterlin, 2001; Oswald and Powdthavee, 2002; Stutzer, 2004; Frey and Stutzer, 2000, 2002) %BIB-FILE:Happiness and Economics: How the Economy and Institutions Affect Human Well-Being)
and should thus offer first insights into the welfare dimension of aspirations-based interventions at the individual level. Closest to this paper, Bernard et al. (2014) report that their role-model treatment did not alter general satisfaction with life. We contribute to this literature by providing experimental evidence of the implications for subjective well-being of shifts in entrepreneurial aspirations. We differ from Bernard et al. (2014) in that we measure both financial and life satisfaction, recognizing the close ties of family and business matters in households with micro-enterprises in the developing world.

Fourth, we contribute to the literature on small-business growth. Dalton et al. (2018) document strong associations between business aspirations and forward-looking firm behavior, such as productive investment, loan take-up, and innovation. Here, we complement this aspirations-based view of entrepreneurial behavior with evidence of both the malleability of business aspirations and their impact on firm profitability. This has implications also for strands of this literature which focus on business mentoring (e.g., Brooks et al., 2018; Cai and Szeidl, 2016), business counseling, consulting, and training (for reviews, see Carpena et al., 2017; McKenzie and Woodruff, 2014), and business plan competitions (e.g., McKenzie, 2017; Bjorvatn et al., 2015). Lastly, our findings speak to a recent literature on the identification of businesses with potential for rapid growth (see Fafchamps and Quinn, 2016; Fafchamps and Woodruff, 2017). We provide evidence on the conditions through which exogenous change in aspirations windows does indeed cause business growth. In particular the heterogeneity we show in treatment effects sheds lights on the importance to know the initial aspiration levels for the design of interventions as for targeting efforts ex ante.

Finally, the paper adds to the growing literature on the effectiveness of role models in promoting behavioral change (see, e.g., Berg and Zia, 2013; Chong and La Ferrara, 2009; Kearney and Levine, 2015; Beaman, et al., 2012; Bernard et al., 2014; Riley, 2017). Emanating from psychology, work on gender stereotypes has shown how exposure to experts and other extraordinary proponents of the same gender  can change stereotypes regarding gender roles and facilitate women's entry into traditionally male industries and roles (e.g., Stout et al., 2011; Asgari et al., 2010). In Economics, interventions involving role models have been used to affect financial knowledge and behaviour (Berg and Zia, 2013), separation and divorce rates (Chong and La Ferrara, 2009), teen pregnancies (Kearney and Levine, 2015), educational outcomes (Beaman et al., 2012; Riley, 2017), or individual investment behavior and savings (Bernard et al., 2014). We add to this in providing evidence that role-model interventions can also affect the growth aspirations of small-business owners and their business performance. We further contribute by quantifying the effect of this intervention against a purely informational shock.


\section{Conceptual Framework} \label{sec:hypotheses}

\subsection{Aspirations and Investment}

The concept of aspirations and its potential for explaining patterns of persistent poverty is not new to the study of development economics. Dalton et al. (2016) develop a model in which differences in initial wealth exacerbate common behavioral biases to produce an aspirations-based poverty trap. In it, behavioral individuals take their aspiration levels as given when choosing effort to invest in the future, even though aspirations are determined by effort and achievement in equilibrium. For both the poor and the rich, this bias leads to suboptimal choices of effort investments. However, since lower wealth levels reduce the marginal benefit of exerting effort, it is the poor individuals who are more likely to aspire below their true potential. That is, poor individuals end up choosing to exert less effort and to set less ambitious aspirations with respect to their true potential. This leads to multiple welfare-ranked equilibria. If constraints to achieve aspirations are not binding and initial aspirations levels are close to an aspirations threshold, an exogenous shock to aspirations can propel the individual out of the aspirations-based poverty trap and move the individual to an equilibrium with higher effort, higher aspirations and better outcomes. Galiani et al. (2018) shed light on the case in which resource constraints are, in fact, binding. Here, the poor individual, once propelled out of the bad equilibrium of a poverty trap through an exogenous shock to aspirations, may not be able to sustain their increased aspirations in the long-term. In the context of a field experiment that randomizes improvements in housing quality to inhabitants of poor slums in Mexico, Uruguay, and El Salvador, the authors show that individuals in the control group indeed report higher aspirations for home improvements in the short-term. However, investment levels did not change and any gains in aspirations have vanished eight months after treatment.

Our paper ties in directly with this literature. In the \emph{Handbook}, we combine step-by-step guidance on easy to implement and locally relevant business practices with the provision of the necessary materials to implement these practices. In so doing, we create an environment in which economic constraints to satisfying increased aspirations can be plausibly assumed not to be binding. In addition, we provide complementary implementation nudges, thought as psychological resources for the entrepreneur to emulate the practices. While the \emph{Movie} offers an opportunity for social learning from peers on the local frontier of best practices, the \emph{Assistance} comprises first-hand experience through personalized hands-on guidance to foster individual agency beliefs. The advice provided in both treatments is based on and almost perfectly equivalent to the content of the \emph{Handbook}, such that treatment effects cannot be driven by a purely informational shock. Since business models in the retail sector, and particularly among small-scale businesses in the traditional sector, are relatively straight-forward and comparable across businesses, we believe this to be a suitable environment to study the impact of opening aspirations windows on small-business growth in the absence of binding economic resource constraints. 

[HYPOTHESIS: TREATMENTS]


%We differ from Galiani et al. (2018) in that we offer not only step-by-step guidance on easy to implement and locally relevant business practices but additionally provide psychological and implementation nudges as resources for the entrepreneur to emulate these practices. That way, we create an environment with essentially zero economic costs to adoption. This is close to Macours and Vakis (2014) who find that, when provided with the necessary resources to follow through on aspired investments, a widening of aspirations windows through exposure to local role models can be sustained beyond the short-term. Furthermore, business models in the retail sector, and particularly among small-scale businesses in the traditional sector, are relatively straight-forward and comparable across businesses. Taken together, we believe this provides us with a realistic case to study the impact of opening aspirations windows on small-business growth when resource constraints are not binding.

\subsection{Aspirations Windows and Aspirations Frustration}

In our approach, we further draw on Ray's (2003, 2006) work on the social formation of aspirations. In this view, individuals have aspirations windows which contain ``her zone of 'similar', 'attainable' individuals" (Ray, 2003, p.1), that is, their ``spatially, economically, perhaps even socially" close others (idem, p.2). Aspirations are formed through ``the lives, achievements, or ideals" (idem, p.2) of such individuals. Consistent with this, there is broad empirical support for the notion that relative status within a community or neighborhood has some bearing on individuals' aspirations (Beaman et al., 2012; Janzen et al., 2017; Knight and Gunatilaka, 2012; Fafchamps and Shilpi, 2008; Stutzer, 2004). In an instructive example, Macours and Vakis (2014) show how plausibly exogenous proximity to females in leadership positions may have opened local women's aspirations windows in a field experiment in Nicaragua.

Inherent in this view of socially determined aspirations is the notion that aspirations may be lifted by opening aspirations windows (see Ray, 2006; Genicot and Ray, 2017; Janzen et al., 2017). Conceptually, aspiration levels are commonly modeled as reference points which affect individual satisfaction derived from achieving a desired goal (see, e.g., Genicot and Ray, 2017; Dalton et al., 2016; Bogliacino and Ortoleva, 2013). In the framework of Genicot and Ray (2017), the agent maximizes the net benefits of effort investment considering two possible outcomes: satisfying their aspirations or failing to satisfy them. Under this scenario, there is a threshold level of aspirations at which the agent is indifferent between exerting high effort and reaping utility from satisfying their aspirations and exerting low effort in frustration. Hence, opening aspirations windows may only be optimal up to a point beyond which individuals in the zone of ``similar others" may become too dissimilar to the agent and aspiration levels too high to be worthwhile attaining. In the words of Ray (2003, p.4): ``If economic betterment is an important goal, the aspirations window must be opened, for otherwise there is no drive to self-betterment. Yet it should not be open too wide: there is the curse of frustrated aspirations."

By opening aspirations windows up to the same level for all entrepreneurs, we ask whether the distance between the individual's aspiration level and the local frontier can predict potential heterogeneity in treatment effects. Entrepreneurs aspiring high at baseline will be more likely to perceive the shock provided through the \emph{Handbook} as a marginal enlargement of their aspirations window by the achievements of similar others and will react by increasing aspiration levels and exerting greater effort. In contrast, entrepreneurs with low aspirations at baseline will see dissimilar others migrate into their aspirations window. As a consequence, these individuals will be more likely to succumb to aspirations failure and will exert less effort. We recognize that purely informational shocks to individuals' aspirations windows may not be sufficient to spur additional effort. That is, in the absence of  economic constraints, psychological constraints to satisfying aspirations may still be present. Hence, in a second step, we test the impact of two kinds of complementary psychological resources that go along with the \emph{Handbook}. First, the \emph{Movie} provides a personal experience with selected entrepreneurs, presented as ``similar'' but successful others. This intervention is aimed at convincing the entrepreneur of the similarity shared between them and provides an opportunity to learn through the actual achievements of a role model rather than to engage with the same set of practices in a purely informational way. Since we designed the advice captured in the \emph{Movie} to be almost perfectly identical with the content of the \emph{Handbook}, any differences between treatment groups are unlikely to be caused by a pure information shock. On the other hand, it is also conceivable that the \emph{Handbook} has limited impact on effort and performance since it makes entrepreneurs engage with material they may construe as being too dissimilar to their own skill set as to engage with it productively. The \emph{Assistance} is designed to remedy this constraint by providing personalized, hands-on assistance which may convince entrepreneurs of the applicability of the selected practices to their own idiosyncratic environment beyond what an informational shock can deliver. 

[HYPOTHESIS: HETEROGENEITY]


%By opening aspirations windows up to the same level for all entrepreneurs, we ask whether the distance between the individual's aspiration level and the local frontier can predict potential heterogeneity in treatment effects. Entrepreneurs aspiring high at baseline will be more likely to perceive the shock provided through the \emph{Handbook} as a marginal enlargement of their aspirations window by the achievements of similar others and will react by increasing aspiration levels and exerting greater effort. In contrast, entrepreneurs with low aspirations at baseline will see dissimilar others migrate into their aspirations window. As a consequence, these individuals will be more likely to succumb to aspirations failure and will exert less effort. We recognize that purely informational shocks to individuals' aspirations windows may not be sufficient to spur additional effort. That is, in the absence of  economic constraints, psychological constraints to satisfying aspirations may still be present. Hence, in a second step, we test the impact of two kinds of complementary psychological resources that go along with the \emph{Handbook}. First, the \emph{Movie} provides a personal experience with selected entrepreneurs, presented as ``similar'' but successful others. This intervention is aimed at convincing the entrepreneur of the similarity shared between them and provides an opportunity to learn through the actual achievements of a role model rather than to engage with the same set of practices in a purely informational way. Since we designed the advice captured in the \emph{Movie} to be almost perfectly identical with the content of the \emph{Handbook}, any differences between treatment groups are unlikely to be caused by a pure information shock. On the other hand, it is also conceivable that the \emph{Handbook} has limited impact on effort and performance since it makes entrepreneurs engage with material they may construe as being too dissimilar to their own skill set as to engage with it productively. The \emph{Assistance} is designed to remedy this constraint by providing personalized, hands-on assistance which may convince entrepreneurs of the applicability of the selected practices to their own idiosyncratic environment beyond what an informational shock can deliver.

\subsection{Aspirations and Welfare}

While the literature on aspirations and poverty has largely established that aspirations correlate with forward-looking behavior and investment (see, e.g., Janzen et al., 2017; Dalton et al., 2018; Kosec and Mo, 2017; Favara, 2017; Ross, 2017; Serneels and Dercon, 2014) and that aspirations are amenable to change (e.g., Bernard et al., 2014; Beaman et al, 2012; McBride, 2010), it is not clear what the welfare consequences of such change should be on the treated individual. As common proxies for individual utility, self-reported happiness and well-being should offer first insights into the impact of aspirations-based interventions on individual welfare (see, e.g., Clark and Oswald, 1994; Oswald, 1997; Ng, 1997; Easterlin, 2001; Oswald and Powdthavee, 2002; Stutzer, 2004; Frey and Stutzer, 2000, 2002). Generally, the happiness literature finds happiness to increase in income but decrease in income aspirations (e.g., Easterlin, 1974, 1995, 2001, 2003; Stutzer, 2004; Knight and Gunatilaka, 2012; for reviews, see Clark et al. 2008; Frey and Stutzer, 2002). Using a large cross-section from Switzerland, Stutzer (2004) provides evidence for a negative correlation between aspiration levels and life satisfaction. Knight and Gunatilaka (2012) find the same result in a cross-section from rural China. McBride (2010) confirms this in the controlled environment of a laboratory study confirming the importance of relative judgments for happiness found in previous lab research (see, e.g., Tversky and Griffin, 1991; Smith et al., 1989). 

The literature offers broad support for two candidate channels which may account for the effect of aspirations on happiness. First, improvements in the incomes of relevant peers tend to decrease individual happiness (e.g., Clark and Senik, 2010; Fafchamps and Shilpi, 2008; Luttmer, 2005; Ferrer-i-Carbonell, 2005; Stutzer, 2004; Senik, 2004, 2009). This social channel is consistent with the literature on the formation of aspirations which models aspiration levels as partly determined by the individual's ``aspirations window'' of similar and attainable peers (Ray, 2006; Genicot and Ray, 2017; Janzen et al., 2017). A shock to the exposure to well-off peers may thus cause changes in the individual's aspiration levels which, in turn, impact on their happiness. 

Second, while most of the literature has been limited to conceptualizing aspirations as one-dimensional and pertaining to income only (see, e.g., McBride, 2010; Stutzer, 2004; Knight and Gunatilaka, 2012), the concept may be inherently multidimensional. This opens the possibility of an internal channel with substitution effects between different aspirations dimensions accounting for the effect. Ray (2003, 2006) acknowledges that aspirations may best be viewed as multi-faceted in nature and that ``depending on one's place in the socio-economic hierarchy, these many-faceted aspirations may complement one another, or they may be mutual substitutes'' (Ray, 2003, p.2). There is some empirical evidence on the interplay of different aspirations dimensions. Considering multiple dimensions of Ethiopian villagers' individual aspirations, Bernard et al. (2014) report treatment effects of a role-model intervention on no other dimension than on educational aspirations and no effects on life satisfaction. While the authors conjecture that the finding may be due to a strong local belief in the returns to education, the aspirations dimensions in question may simply act as substitutes rather than complements. In that view, an exogenous shock may render one dimension salient at the expense of another in the pursuit of a goal or change the relative marginal benefits. Bjorvatn et al. (2015) offers further suggestive evidence along these lines from a field experiment among school students in Tanzania. The authors show that exposure to an edutainment program that motivated students for entrepreneurship facilitated interest in entrepreneurship and increased rates of business start-up. However, by the same token, treatment decreased school performance and graduation rates.

In this study, we are able to shed light on both of these channels and measure the impact of aspirations on subjective well-being. We investigate the social channel in that we expose entrepreneurs to best practices among their local peers in the \emph{Handbook} and by doubling down on the effect social learning with the \emph{Movie}. We consider the multidimensionality of aspirations by measuring the entrepreneur's aspirations for their children's educational attainments alongside their business aspirations. Impact on subjective well-being is captured by two questions about the satisfaction of the entrepreneur with their financial situation and with life in general.

[HYPOTHESIS: OTHER ASPIRATIONS, SWB]



%Second, it may be the inherent multidimensionality of aspirations that accounts for the effect and the possibility of substitution effects between different aspirations dimensions. Acknowledging this internal channel, Ray (2003, 2006) argues that aspirations may best be viewed as multi-faceted in nature and that ``depending on one's place in the socio-economic hierarchy, these many-faceted aspirations may complement one another, or they may be mutual substitutes'' (Ray, 2003, p.2). While most of the literature has been limited to conceptualizing aspirations as one-dimensional and as pertaining to income only (see, e.g., McBride, 2010; Stutzer, 2004; Knight and Gunatilaka, 2012), there is some recent evidence on the interplay of different aspirations dimensions. Considering multiple dimensions of Ethiopian villagers' individual aspirations, Bernard et al. (2014) report treatment effects of a role-model intervention on no other dimension than on educational aspirations and no effects on life satisfaction. While the authors conjecture that the finding may be due to a strong local belief in the returns to education, the aspirations dimensions in question may simply act as substitutes rather than complements. In that view, an exogenous shock may render one dimension salient at the expense of another in the pursuit of a goal or change the relative marginal benefits. Bjorvatn et al. (2015) offers further suggestive evidence along these lines from a field experiment among school students in Tanzania. The authors show that exposure to an edutainment program that motivated students for entrepreneurship facilitated interest in entrepreneurship and increased rates of business start-up. However, by the same token, it decreased school performance and graduation rates.


%It is not clear whether and, if so, how different dimensions of aspirations are traded off against each other. This also renders the welfare effects of aspirations-based interventions potentially ambiguous: if positive exogenous change to one dimension can trigger unintended consequences on other dimensions, it is not clear whether the recipient, as a consequence, will be better or worse off. 
%While Bernard et al. (2014) report that their role-model treatment did not alter general satisfaction with life. We contribute to this literature by providing experimental evidence of the implications for subjective well-being of shifts in entrepreneurial aspirations. We differ from Bernard et al. (2014) in that we measure both financial and life satisfaction, recognizing the close ties of family and business matters in households with micro-enterprises in the developing world.

\section{Results}\label{sec.analysis}

\subsection{Impact on Business Aspirations}

Tables A and B show the heterogeneity in treatment effects on business aspirations for subgroups of entrepreneurs with above- and below-median baseline aspirations six months and 18 months after treatment, respectively. %Table A shows results six months after treatment and Table B reports findings from the 18-months endline.
For each, Columns (1) to (5) present results on short-run aspirations for the business in 12 months and Columns (6) to (9) present results on long-run aspirations for the entrepreneur's ideal business. Overall, we find a pronounced divergence by which business owners with higher aspirations at baseline increase their aspirations at endline and those who aspired low decrease their aspirations further. While effects are modest six months after the interventions (Table A), they grow more pronounced eighteen months out (Table B).

As can be seen in Table A, six months after the interventions, treatment effects on business aspirations are (i) more pronounced for shop owners with below-median baseline aspirations than for those with above-median aspirations and they are (ii) stronger for long-run aspirations than for short-run aspirations. Column (6) shows that, on aggregate, entrepreneurs who start out with lower aspiration levels become more discouraged in their long-run aspirations after being assigned to the \emph{Handbook} and to \emph{All Three} whereas those with higher aspirations to begin with do not. %F-tests reveal that low-aspiring shop owners' aspirations also falter when assigned to \emph{All Three}.
This aggregate effect is mostly driven by long-run aspirations for business size. According to Column (7), entrepreneurs below the median of business aspirations at baseline aspire for businesses about 4.5 (-22\%) and 6.9 square meters (-34\%) smaller in size after being assigned to \emph{Handbook and Assistance} and \emph{All Three}, respectively. In contrast, the aspirations of entrepreneurs with high aspirations are not statistically different from those in the control group. Changes in short-run aspirations are more modest and mostly not statistically significant. \\ %As per Column (4), entrepreneurs aspiring higher than the median business at baseline aspire for around 20 more daily customers within 12 months of being assigned to \emph{Handbook and Assistance}}.

The divergence in aspiration levels grows stronger eighteen month after treatment. As Table B shows, significant effects can now be (i) observed for both entrepreneurs with below-median and those with above-median baseline aspirations, for (ii) both short-run and long-run aspirations, and they are (iii) most pronounced for customer and sales aspirations. Column (4) shows that, being assigned to the \emph{Movie}, entrepreneurs with above-median customer aspirations at baseline increase their aspirations by about 20 daily customers (+29\% or 0.37 standard deviations) compared to the control group. In addition, entrepreneurs invited to receive \emph{Assistance} and \emph{All Three} each aspire to roughly 15 extra customers over the control (+21\% or 0.27 standard deviations). In contrast, shop owners with below-median aspirations further lower their aspirations by about 17 daily customers (-43\% or 0.57 standard deviations) vis-\`{a}-vis the control when assigned to the \emph{Handbook Only}, and by 8 (-21\%) and 9 customers (-22\%) when additionally invited to the \emph{Movie} and \emph{Assistance}, respectively. Column (5) shows that assignment to the \emph{Movie} also increases the sales aspirations of high-aspiring entrepreneurs. While this subgroup aspires to a considerable plus of USD 151.00 PPP in daily sales (+20\% or 0.23 standard deviations), low-aspiring shop owners aspire to about USD 67.00 PPP less after the same intervention (-9\% or 0.10 standard deviations). Shop owners with low baseline aspirations also decrease their sales aspirations when assigned to the \emph{Handbook Only} and to \emph{All Three}.

Long-run aspirations corroborate this pattern. The negative aggregate effect for shop owners who aspire below the median at baseline, shown in Column (6), is driven by aspirations for business size (Column 7) and customers (Column 9). As per Column (9), shop owners who have high customer aspirations at baseline increase their aspirations by 25 (+29\%) and 36 daily customers (+41\%) over the control when assigned to \emph{Handbook Only} and \emph{Handbook and Movie}, respectively. On the contrary, the \emph{Handbook} lowers the customer aspirations of entrepreneurs with low aspirations at baseline by about 20 customers (-34\% or 0.27 standard deviations) compared to the control group. Column (7) shows no significant positive effects on long-run aspirations for business size. However, all interventions significantly reduce these aspirations for shop owners with below-median baseline aspirations. Effects for \emph{Handbook Only}, \emph{Handbook and Movie}, and \emph{Handbook and Assistance} are considerable: low-aspiring entrepreneurs reduce their aspirations vis-\`{a}-vis the control group by about 6 (-30\%), 5 (-25\%), and 5 (-22\%) square meters, respectively.


\subsection{Impact on Business Performance}

Tables C and D present results on treatment effects on business profits and sales for the same subgroups of high- and low-aspiring entrepreneurs. In each case, Columns (1) and (2) present results on calculated monthly profits and Columns (3) and (4) present results on monthly business sales. Columns (2) and (4) show estimates for the respective outcome winsorized at the 1\% level on both tails. Overall, the pattern we observe matches the divergence found for business aspirations: the interventions spur higher sales and profits for business owners with high aspirations at baseline and lower performance for their low-aspiring peers. We find strong effects in terms of statistical and economic significance both six months (Table C) and 18 months after treatment (Table D). Interestingly, while entrepreneurs above the median gain in performance from both the \emph{Movie} and the \emph{Assistance} but not from the \emph{Handbook Only}, the aspirations of below-median entrepreneurs only falter when assigned to the \emph{Handbook Only} but not when invited to the \emph{Movie} or the \emph{Assistance}.

Table C shows significant and positive treatment effects on business profits and sales for entrepreneurs with above-median baseline aspirations. As per Column (3), high-aspiring shop owners report a large increase of about USD 578.00 PPP (+47\% or 0.28 standard deviations) over the control group when invited to receive \emph{Assistance}. Column (6) shows that these same entrepreneurs also report gains in monthly business sales of roughly USD 1598.00 PPP (+18\% or 0.17 standard deviations). Entrepreneurs with high baseline levels of aspirations who are assigned to the \emph{Movie} report gains in monthly profits of about USD 405.00 PPP (+33\% or 0.20 standard deviations) and in monthly sales of roughly USD 1329.00 PPP over the control (+15\% or 0.14 standard deviations). Shop owners with high baseline aspirations assigned to \emph{All Three} report increases in monthly profits and sales of about USD 489.00 PP (+40\%) and USD 1611.00 PPP (+18\%), respectively, which indicates that we find no complentarities between the interventions. While entrepreneurs with below-median aspirations report lower profits and sales, F-tests reveal that most estimates are not significantly different from the control group. Only when distributed the \emph{Handbook Only} do low-aspiring entrepreneurs report significant reductions in monthly business sales by USD 1088.00 PPP (-41\% or 0.54 standard deviations) without significant decreases in monthly profits.

Table D corroborates this divergent trend with data from the 18-months endline survey. Overall, it becomes clear that the most robust positive effects on business profits and sales accrue to those assigned to \emph{All Three}. High-aspiring entrepreneurs report monthly gains in profits of USD 623.00 PPP (x\% or 0.xx standard devaitions) and in sales of USD 1797.00 PPP (x\% or 0.xx standard deviations) over the control group, which is roughly on par with results from the first endline. Column (3) shows that we find significant effects of the \emph{Assistance} only when we winsorize 2\% of the outcome data on both tails. Here, entrepreneurs with high initial aspiration levels report USD 526.00 PPP higher profits (x\% or 0.xx standard deviations).

\subsection{Impact on Aspirations for Children's Education}

In Table E, we present additional findings on the heterogeneity in treatment effects on aspirations for the education of the entrepreneur's children six months after treatment. While Columns (1) and (2) present estimates for an aggregate score representing the statistical average of aspirations towards both the family's son and the family's daughter, Columns (3) to (4) show estimates regarding the family's oldest son under the age of 18, and Columns (5) to (6) show estimates regarding the daughter. We find evidence for the same heterogeneity in impact across subgroups that we also report for business aspirations and business performance. That is, as per Column (1), entrepreneurs with above-median baseline aspirations report to aspire to almost one year (5\% or 0.30 standard deviations) more educational attainment for their children than the control group. Similarly, according to Column (2), they are roughly 13\% more likely to aspire for their children to reach, on average, Masters-level university education. Evidence of negative effects on low-aspiring entrepreneurs is less conclusive. Only aspirations for the son's education show a significant and negative effect for entrepreneurs invited to \emph{All Three}: this subgroup aspires to 1 years (6\% or 0.30 standard deviations) less in schooling compared to the control group.

\subsection{Impact on Subjective Well-Being}

Table F presents significant and sustained positive effects on overall financial satisfaction for entrepreneurs with high baseline aspirations both six months (Column 2) and 18 months after treatment (Column 4). High-aspiring shop owners at baseline have gained 9\% (0.26 standard deviations) in satisfaction after six months and report a plus of X\% (0.XX standard deviations) over the control group 18 months after treatment. [COMPLETE WHEN CONTROL STATS CORRECTED] Moreover, we detect a significant increase in life satisfaction scores of X\% (0.xx standard deviations) for these same entrepreneurs 18 months after the treatment (Table G, Column 1). The satisfaction scores of entrepreneurs with below-median baseline aspirations are not significantly different from the control group.

\subsection{Characteristics of High-Aspiring Entrepreneurs}

Given the clear, robust, and sustained divergence between entrepreneurs with high and low aspirations at baseline, we investigate the determinants of high aspirations. To this end, we use baseline data on the full sample of entrepreneurs to regress a pair of dummies indicating above-median aspirations for business growth in the short- and in the long-run. Table H presents results from these linear OLS regressions. Columns (1) to (5) take a dummy for above-median short-term aspirations as the outcome, Columns (6) to (10) use a dummy for above-median long-term aspirations. We consecutively add vectors of entrepreneur-level and firm-level characteristics and control for village fixed effects. Columns (5) and (10) show that several characteristics stand out as determining above-median aspirations for both the short- and for the long-term. Businesses whose owners report high aspirations are in existence for a shorter time, they are larger in terms of both business size and the number of employees, and operated by entrepreneurs who are younger and more likely to be male. Entrepreneurs who aspire high are more likely to produce product innovations and are more systematic in their approach to working and thinking. In addition, these entrepreneurs are better skilled at keeping records and planning ahead financially as well as more willing to take risks overall.

%\subsection{Robustness Tests}

%Corrections for Multiple Hypothesis Testing? Anderson (2008)


\end{document}